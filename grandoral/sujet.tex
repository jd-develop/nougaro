% This document is placed under the GNU FDL license

\documentclass{article}
\usepackage{indentfirst}
\usepackage[french]{babel}
\usepackage[T1]{fontenc}
\usepackage{hyperref}
\usepackage[sorting=none]{biblatex}
\usepackage{csquotes}
\usepackage{geometry}
\usepackage{listings}
\addbibresource{bibliography.bib}

\geometry{a4paper,left=2.5cm,right=2.5cm,top=2cm}

\title{Comment fonctionne l’interpréteur d’un langage de programmation\nobreakspace?}
\author{Jean Dubois}

\begin{document}

\maketitle

\section{Introduction}
Les ordinateurs sont des machines traitant des données et pouvant être programmées.
On estime que le premier vrai algorithme a été écrit pour la machine analytique imaginée
en 1834 par Charles Babbage, par la première informaticienne de l’humanité\nobreakspace:
Ada Lovelace. Cette machine ne fut cependant jamais construite. Les programmes écrits par
Ada Lovelace étaient rédigés en langage mathématique, et non dans un langage de programmation
tel qu’on en connaît aujourd’hui. Cependant, elle fut la première à comprendre la différence
entre matériel et logiciel. \cite[p. 454]{berry}

Les premiers ordinateurs électroniques, tel que l’ENIAC (Electronic Numerical Integrator and Computer),
se programmait en connectant physiquement les différents modules entre eux. \cite{britannica-eniac}
Après cela, les ordinateurs ont progressivement adoptés un langage machine, où chaque instruction est
représentée par un nombre, puis un langage assembleur, qui fait correspondre chaque instruction à
un mot-clé mnémotechnique. \cite{britannica-asm}

Le premier vrai langage de programmation à proprement parler est Fortran, pour mathematical
FORmula TRANslating system, créé en 1954 pour le calcul scientifique \cite[p. 455]{berry}.
Fortran était un langage compilé et haut-niveau, qui essayait au mieux de reproduire la
notation mathématique. Jusqu’alors, un programme était une suite d’instructions pas-à-pas pour
le processeur. En Fortran, l’idée était plutôt de décrire le problème que l’on veut résoudre 
\cite{imb-john-backus}, et cette idée va être à l’origine de tous les autres langages de
programmations jusqu’à nos jours.

Fortran est ensuite suivi par LISP (LISt Processor), qui est le premier langage interprété.

Nous allons nous intéresser dans cet exposé au fonctionnement des langages interprétés,
bien que nous parlerons également des langages compilés. Nous expliciterons d’abord les
points communs et les différences entre ces deux types de langage. Nous verrons ensuite
dans le détail comment un interpréteur fonctionne. Enfin, nous nous intéresserons à
comment implémenter un interpréteur en langage Python, et notamment en nous intéressant
à l’interpréteur du langage Nougaro, écrit par mes soins.

\section{Langages interprétés et langages compilés}
Tout d’abord, qu’est-ce qu’un langage de programmation\nobreakspace?
Un langage de programmation permet de communiquer des instructions à un ordinateur.
Comme on l’a déjà discuté, les ordinateurs ne parlent qu’en langage machine, dans lequel
on parle en cases mémoires et instructions basiques. Dans un langage de programmation,
les instructions sont plus abstraites, plus proches de comment pense un être humain. \cite[p. 55]{berry}

Cependant, il faut traduire ce langage de programmation en langage machine. C’est le rôle
des compilateurs. Un compilateur prend en entrée un programme écrit dans un certain langage
et le traduit dans un autre langage. \cite{github-programming-languages}
À l’origine, cet autre langage était le langage machine,
mais de plus en plus de compilateurs traduisent maintenant vers le C, qui a le mérite de posséder
des dizaines de compilateurs stables vers le langage machine, et d’être un des langages les
plus utilisés sur Terre.

On peut remarquer que la plupart des compilateurs sont écrits dans le
langage qu’ils sont censés interpréter (\textit{GCC} est écrit en C, \textit{RustCompiler}
est écrit en Rust, etc.)
Cela peut paraître paradoxal, mais a en fait une explication\nobreakspace: il suffit de programmer
le compilateur avec un certain langage, et d’écrire une version très simple de notre langage.
Puis, on utilise notre compilateur pour le faire lui-même se compiler. On peut ensuite utiliser notre
langage pour écrire notre vrai compilateur. On compile ce programme grâce à notre premier compilateur
puis on le recompile avec le résultat de la compilation. Ce dernier compilateur ne contient plus
aucune trace du langage originel ni des étapes intermédiaires. Ce principe se nomme le
\textit{bootstraping}. \cite[p. 448]{berry}

Un interpréteur, quant à lui, est un programme qui exécute pas-à-pas les instructions indiquées
dans le programme. Un interpréteur ne traduit pas vraiment vers un autre langage, mais interprète
ce que l’utilisateur a rentré. \cite{github-programming-languages}
Nous ne nous intéresserons désormais qu’aux interpréteurs.

\section{Fonctionnement détaillé d’un interpréteur}
Il y a trois grandes façons d’interpréter un programme\nobreakspace: analyser directement le code
et l’exécuter dans la foulée, transformer le code vers une représentation intermédiaire et exécuter
cette représentation, et compiler vers un langage intermédiaire plus simple et interpréter ce
résultat. \cite{wikipedia-interpreter} Je m’intéresserai à la seconde méthode.

Un interpréteur est constitué de trois parties distinctes, chacune s’exécutant l’une après l’autre.
Ces trois parties sont l’analyseur lexical (lexer), l’analyseur syntaxique (parser), et l’interpréteur
à proprement parler.

\subsection{Analyse lexicale}
L’analyse lexicale a pour but de découper le code source en petits bouts, des 
\guillemetleft\nobreakspace tokens\nobreakspace\guillemetright\nobreakspace en anglais,
que l’on pourrait traduire en
\guillemetleft\nobreakspace unités\nobreakspace\guillemetright\nobreakspace.

Une unité est composée d’un type, d’une position, et éventuellement d’une valeur.
Ces unités peuvent être de plusieurs types\nobreakspace: entier, flottant, chaîne de caractères,
signe \verb|+|, signe \verb|=|, mot-clé, symbole (identifiant), etc.

Par exemple, le code suivant\nobreakspace:

\begin{lstlisting}
var a = 0
for i = 0 to 10 then
    var a += i
    print(a)
end
\end{lstlisting}

sera traduit par l’analyseur lexical vers la liste d’unités suivante\nobreakspace:

\begin{lstlisting}
keyword:var, identifier:a, eq, int:0, newline,
keyword:for, identifier:i, eq, int:0, keyword:to, int:10, keyword:then,
    newline
keyword:var, identifier:a, pluseq, identifier:i, newline,
identifier:print, lparen, identifier:a, rparen, newline,
keyword:end, end-of-file
\end{lstlisting}

L’analyseur lexical repère les identifiants (\verb|identifier|), et les distingue des
des mot-clés (\verb|keyword|) grâce à une liste. Les symboles \verb|=| et \verb|+=| sont
traduits par les unités \verb|eq| et \verb|pluseq|.

Il est important de noter que pour l’analyseur lexical, la notion de syntaxe n’existe pas encore.
Par exemple, lui donner \verb|while for True in print 78.5 += "chaîne"| ne sortira pas d’erreur,
et on aura la liste d’unités correspondante en sortie. Il n’y a pas non plus encore de notion de
contexte et de variables définies ou non, il ne donnera donc pas d’erreur si un identifiant n’est
pas défini.

La liste d’unités que l’analyseur lexical donnera en sortie sera donnée en entrée à l’étape
suivante\nobreakspace: l’analyseur syntaxique (ou parser).

\subsection{Analyse syntaxique}
L’analyse syntaxique a pour but de construire un arbre de syntaxe (en anglais Abstract Syntax
Tree, AST), constitué de nœuds, à partir des unités de l’analyseur lexical et d’une grammaire
définie.

Pour cela, il parcourt les unités une à une et repère les structures.
Le but est, pour chaque unité, de parcourir la grammaire qu’on s’est définie.

Par exemple, avec notre programme de tout à l’heure, on trouve d’abord un mot-clé \verb|var|. 
On sait qu’on est dans une déclaration de variable, on sait alors que l’unité suivante doit
impérativement être un identifiant. Si ce n’est pas le cas, on renvoie une erreur.
Ici, on trouve un identifiant, de valeur \verb|a|. On doit ensuite trouver un signe \verb|=|,
puis une valeur. La valeur peut être n’importe quelle expression, ici il s’agit simplement d’un
entier\nobreakspace: 0. On s’attend ensuite à trouver un retour-ligne, car cette instruction est terminée.
Une fois qu’on a fait ça, on ajoute à notre liste de nœuds un nœud de déclaration de variable. Ce
nœud possède une position, un nom de variable, une valeur.

On continue à parcourir les unités, en on trouve le mot-clé \verb|for|. On sait qu’on doit trouver
ensuite un identifiant, puis il y a deux syntaxes (\verb|for i in| et \verb|for i = to|). Ici, on trouve
un signe \verb|=|, on sait qu’on est dans la deuxième syntaxe. On continue donc jusqu’au \verb|then|,
puis au retour-ligne. À partir de là, on ajoute chaque instruction présente dans la boucle à une liste
de nœuds, qui sera elle-même stockée dans le nœud de la boucle \verb|for|. On continue chaque instruction
jusqu’au \verb|end|.

Une chose intéressante est l’appel de fonction\nobreakspace: on ne sait pas si c’est un simple accès à une
variable ou un appel avant d’avoir trouvé l’unité \verb|lparen| correspondant à la parenthèse ouvrante.

C’est donc l’analyseur syntaxique qui s’occupe de savoir si la grammaire est respectée. Dans notre
exemple \verb|while for True in print 78.5 += "chaîne"|, l’analyseur syntaxique
ne s’attendrait pas à trouver le \verb|for| après le \verb|while|, et renverrait donc une erreur.
En revanche, l’analyseur syntaxique ne s’occupe toujours pas de si une variable est définie ou non.
Ainsi, si l’on n’avait pas définit la variable \verb|a| au début de notre code, l’analyseur syntaxique
n’aurait pas renvoyé d’erreur.

À la fin de l’analyse syntaxique, on se retrouve avec une liste de nœuds, que l’on peut envoyer
vers l’interpréteur, qui va s’occuper d’exécuter réellement le code.

\subsection{Interpréteur}
L’interpréteur, en anglais \textit{interpreter} ou plus spécifiquement \textit{runtime}, est la dernière
étape, celle qui exécute l’arbre de syntaxe directement.
L’interpréteur stocke un contexte, qui contient plusieurs informations utiles\nobreakspace:
les variables dans un dictionnaire identifiant/valeurs, ou encore s’il faut arrêter ou continuer
une boucle au cas où on aurait trouvé un mot-clé \verb|continue| ou \verb|break|.

L’interpréteur parcourt l’arbre de syntaxe, et exécute chaque nœud. Par exemple, s’il trouve un
nœud \verb|CallNode| d’identifiant \verb|print| et d’argument \verb|a|, il va résoudre le nom
\verb|a| dans sa table de symboles, puis va résoudre le nom \verb|print|, qui va afficher la 
valeur de \verb|a| à l’écran.

\section{Une implémentation en Python}
(détailler le jour de l’oral si le temps)

(parler des limites\nobreakspace: Python n’est pas fait pour de la production et surtout c’est
très lent)

\printbibliography

\section*{Licence}
Ce document est placé sous licence GNU Free documentation license (FDL).
Cela signifie que quiconque peut le copier et le redistribuer librement, avec ou
sans modifications, dans un but commercial ou non, à condition de placer le
document copié et/ou redistribué sous licence GNU FDL ou sous une licence
compatible.

Plus précisément, ce document est placé sous les termes de la GNU FDL version 1.3 ou
ultérieure, disponible en ligne à l’adresse
\href{www.gnu.org/licenses/fdl-1.3.html}{www.gnu.org/licenses/fdl-1.3.html}.
Une copie du document de licence est normalement distribué avec ce document.
Si ce document venait à être redistribué au format PDF, l’auteur apprécierait
particulièrement la redistribution du code \LaTeX\nobreakspace en plus du PDF.

\end{document}
